%%%%%%%%%%%%%%%%% ASYLUM REBOOT

% ASYLUM reboot is an adaptation of the original ASYLUM card game.
% Its content belongs to the creator and should'nt be printed, diffused or commercialized without its author permission.

%Renaud Dubois: creator and game designer, artistic lead


\documentclass[parskip]{scrartcl}
\usepackage[utf8]{inputenc}
\usepackage[T1]{fontenc}
\usepackage[paperheight=96mm,paperwidth=71mm,margin=0mm]{geometry}
\usepackage{xintexpr}
\usepackage{tikz}
\usetikzlibrary{positioning}
\usetikzlibrary{shapes.arrows}
\usetikzlibrary{fadings}
\usepackage{pifont}
\usepackage{graphicx}
\usepackage{xfp}
\usepackage{ifthen}
\usepackage{atbegshi}
\AtBeginDocument{\AtBeginShipoutNext{\AtBeginShipoutDiscard}} %pour supprimer la premiere page

\DeclareOldFontCommand{\bf}{\normalfont\bfseries}{\mathbf}

\begin{document}

\pgfmathsetmacro{\cardroundingradius}{4mm}
\pgfmathsetmacro{\cardwidth}{6.8}
\pgfmathsetmacro{\cardheight}{9.3}

\pgfmathsetmacro{\titleX}{3.55}
\pgfmathsetmacro{\titleY}{8.7}
\pgfmathsetmacro{\imageX}{3.555}


\pgfmathsetmacro{\imageY}{6.32}
\pgfmathsetmacro{\typeX}{3.55}
\pgfmathsetmacro{\typeY}{4}
\pgfmathsetmacro{\imageWidth}{159}
\pgfmathsetmacro{\imageHeight}{115}
\pgfmathsetmacro{\descriptionX}{.63}
\pgfmathsetmacro{\descriptionY}{3.7}
%\pgfmathsetmacro{\punchlineX}{.5}
%\pgfmathsetmacro{\punchlineY}{1.8}
\pgfmathsetmacro{\numberX}{5.8}
\pgfmathsetmacro{\numberY}{.94}

\renewcommand{\titlefont}{}
\newcommand{\typefont}{\bf}
\newcommand{\punchlinefont}{\itshape}
\newcommand{\descriptionfont}{}
\newcommand{\numberfont}{}
\newcommand{\setsize}[1]{\fontsize{#1pt}{\fpeval{1*(#1)}pt}\selectfont}

\newcounter{cardctr}
\newcommand{\cardnumber}{\stepcounter{cardctr}\thecardctr\ifthenelse{\equal{\thecardctr}{8}}{\setcounter{cardctr}{0}}{}}

%VersoPerso
\newcommand{\versoperso}{ %TODO A recentrer et elargir
\begin{tikzpicture}
    \node[anchor=south west,inner sep=0] at (0,0) {\includegraphics[width=7.1 cm, height=9.6 cm]{fonds/versoperso.pdf}};
\end{tikzpicture}}

%Verso
\newcommand{\verso}{ %TODO A recentrer et elargir
\begin{tikzpicture}
    \node[anchor=south west,inner sep=0] at (0,0) {\includegraphics[width=7.1 cm, height=9.6 cm]{fonds/back.pdf}};
\end{tikzpicture}}


%DEBUT DU DOCUMENT

%--------------------------CARTES NEUTRES-----------------------------------------------------------------------------------------------
%!TEX root = lot2.tex

%--------------------------CARTES NEUTRE------------------------------------------------------------------------------------------------

\begin{tikzpicture} %Recto
	%Fond
    \node[anchor=south west,inner sep=0] (carte) at (0,0) {\includegraphics[width=7.1 cm, height=9.6 cm]{fonds/noir.png}};
    \node[anchor=center] at (carte.center) {\includegraphics[width=\cardwidth cm, height=\cardheight cm]{fonds/fond_plan.png}};
\end{tikzpicture}

\begin{tikzpicture} %Verso
	%Fond
    \node[anchor=south west,inner sep=0] (carte) at (0,0) {\includegraphics[width=7.1 cm, height=9.6 cm]{fonds/noir.png}};
    \node[anchor=center] at (carte.center) {\includegraphics[width=\cardwidth cm, height=\cardheight cm]{fonds/fond_plan.png}};
\end{tikzpicture}

%%%%%%%%%%%%%%%%%%%%%%%%%%%%%%%%%%%%%%%%%%NEVER MISS A DROP
\begin{tikzpicture} %Recto
	%Fond
    \node[anchor=south west,inner sep=0] (carte) at (0,0) {\includegraphics[width=7.1 cm, height=9.6 cm]{fonds/noir.png}};
    \node[anchor=center] at (carte.center) {\includegraphics[width=\cardwidth cm, height=\cardheight cm]{fonds/fond_neutre.png}};

    %Titre
	\node[anchor=center] at (\titleX,\titleY) {\titlefont Never Miss A Drop !};

	%Image
	\node[anchor=center] at (\imageX,\imageY) {\includegraphics[width=\imageWidth px, height=\imageHeight px]{images/never_miss.jpg}};
	\node[anchor=center] at (6.1,4.5) {\includegraphics[width=12 px, height=6 px]{fonds2/legacy.jpg}};

	%Type
	\node[anchor=center] at (\typeX,\typeY) {\typefont Neutre};

	%Description
	\node[anchor=north west, text width=5.6cm] (description) at (\descriptionX,\descriptionY) {\descriptionfont\setsize{7} Grace au market piss, ne ratez plus une goutte ! Remplissez un verre d'eau à la hauteur de votre choix. Jusqu'à ce qu'un joueur déborde, chacun doit ajouter un NFT (faire couler de l'eau) dans le verre. Celui qui échoue doit piocher une carte, puis le tour se termine\par};

	%Punchline
	\node[anchor=north west, text width=5.6cm, below = 1pt of description] (punchline) {\punchlinefont\setsize{7}``Pour plus d'inclusivité, les toilettes d'entresol sont désormais agenrées.''\par};

	%Separateur !!!!!PAS TOUCHE!!!!!
	\fill[black,path fading=west] (description.south west) rectangle (punchline.north);
	\fill[black,path fading=east] (punchline.north) rectangle (description.south east);

	%Numéro !!!!!PAS TOUCHE!!!!!
	\node[anchor=center] at (\numberX,\numberY) {\numberfont \cardnumber};
\end{tikzpicture}\verso %Verso



%%%%%%%%%%%%%%%%%%%%%%%%%%%%%%%%%%%%%%%%%%HAN SOLO ROGERS
\begin{tikzpicture} %Recto
	%Fond
    \node[anchor=south west,inner sep=0] (carte) at (0,0) {\includegraphics[width=7.1 cm, height=9.6 cm]{fonds/noir.png}};
    \node[anchor=center] at (carte.center) {\includegraphics[width=\cardwidth cm, height=\cardheight cm]{fonds/fond_neutre.png}};

    %Titre
	\node[anchor=center] at (\titleX,\titleY) {\titlefont Han Solo Rogers};

	%Image
	\node[anchor=center] at (\imageX,\imageY) {\includegraphics[width=\imageWidth px, height=\imageHeight px]{images/ian.jpg}};
	\node[anchor=center] at (6.1,4.5) {\includegraphics[width=12 px, height=6 px]{fonds2/legacy.jpg}};

	%Type
	\node[anchor=center] at (\typeX,\typeY) {\typefont Neutre};

	%Description
	\node[anchor=north west, text width=5.6cm] (description) at (\descriptionX,\descriptionY) {\descriptionfont\setsize{7} Vous allez dégoter le futur artiste qui sera sponsorisé par le Market Pass ! Chaque joueur doit faire un dessin de singe. Vous choisissez le plus beau, l'auteur peut défausser une carte.\par};

	%Punchline
	\node[anchor=north west, text width=5.6cm, below = 1pt of description] (punchline) {\punchlinefont\setsize{7}``I have got a very good feeling about this.''\par};

	%Separateur !!!!!PAS TOUCHE!!!!!
	\fill[black,path fading=west] (description.south west) rectangle (punchline.north);
	\fill[black,path fading=east] (punchline.north) rectangle (description.south east);

	%Numéro !!!!!PAS TOUCHE!!!!!
	\node[anchor=center] at (\numberX,\numberY) {\numberfont \cardnumber};
\end{tikzpicture}\verso %Verso








\end{document}
